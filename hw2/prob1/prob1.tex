\documentclass[10pt]{article}
\usepackage[a4paper]{geometry}
\usepackage[utf8]{inputenc}
\usepackage{fullpage}
\usepackage{amssymb}
\usepackage{amsmath}


\linespread{1.18}
\pagenumbering{gobble}

%%%%%%%%%%%%%%%%%%%%%%%%%
%%%% Title & Heading %%%%
%%%%%%%%%%%%%%%%%%%%%%%%%

\title{\huge\textbf{HW2: Problem 1}}
\author{\textit{Mehran Shakerinava}}
\date{October 2019}

\begin{document}

\maketitle

\noindent
For brevity, I'll write $\mathbb{E}_{x\sim p_\theta (X|y)}$ as $\mathbb{E}_{x|y}$.\\
\textit{The following result forms the basis of the solutions to this problem:}

\begin{equation*}
\begin{aligned}
\mathbb{E}_{a_t|s_t} \left[ \nabla_\theta \log \pi_\theta(a_t|s_t) \right] &= \int \pi_\theta(a_t|s_t) \nabla_\theta \log \pi_\theta(a_t|s_t) da_t \\
&= \int \pi_\theta(a_t|s_t) \frac {\nabla_\theta \pi_\theta(a_t|s_t)}{\pi_\theta(a_t|s_t)} da_t \\
&= \int \nabla_\theta \pi_\theta(a_t|s_t) da_t \\
&= \nabla_\theta \int \pi_\theta(a_t|s_t) da_t \\
&= \nabla_\theta 1 \\
&= 0
\end{aligned}
\end{equation*}


\section*{Part A}

\begin{equation*}
\begin{aligned}
\mathbb{E}_{\tau}\left[ \nabla_\theta \log \pi_\theta(a_t|s_t) b(s_t) \right] &= \mathbb{E}_{s_t, a_t} \left[ \mathbb{E}_{\tau/s_t, a_t|s_t, a_t} \left[ \nabla_\theta \log \pi_\theta(a_t|s_t) b(s_t) \right] \right] \\
&= \mathbb{E}_{s_t, a_t} \left[ \nabla_\theta \log \pi_\theta(a_t|s_t) b(s_t) \right] \\
&= \mathbb{E}_{s_t} \left[ \mathbb{E}_{a_t|s_t} \left[ \nabla_\theta \log \pi_\theta(a_t|s_t) b(s_t) \right] \right] \\
&= \mathbb{E}_{s_t} \left[ b(s_t) \mathbb{E}_{a_t|s_t} \left[ \nabla_\theta \log \pi_\theta(a_t|s_t) \right] \right] \\
&= 0
\end{aligned}
\end{equation*}


\section*{Part B}

\subsection*{a)}
Because of the Markov property, given $s_t$, the distribution of states and actions after time $t$ is independent of states and actions before time $t$. This implies that $\nabla_\theta \log \pi_\theta(a_t|s_t) b(s_t)$, which is a function of $s_t$ and $a_t$, is independent of $(s_1, a_1, ..., a_{t-1})$ given $s_t$, and thus, conditioning on $(s_1, a_1, ..., a_{t-1}, s_t)$ is equivalent to conditioning only on $s_t$.

\subsection*{b)}
\begin{equation*}
\begin{aligned}
\mathbb{E}_{\tau}\left[ \nabla_\theta \log \pi_\theta(a_t|s_t) b(s_t) \right] &= \mathbb{E}_{s_{1:t}, a_{1:t-1}} \left[ \mathbb{E}_{s_{t+1:T}, a_{t:T} | s_{1:t}, a_{1:t-1}} \left[ \nabla_\theta \log \pi_\theta(a_t|s_t) b(s_t) \right] \right] \\
&= \mathbb{E}_{s_{1:t}, a_{1:t-1}} \left[ \mathbb{E}_{s_{t+1:T}, a_{t:T} | s_t} \left[ \nabla_\theta \log \pi_\theta(a_t|s_t) b(s_t) \right] \right] \\
&= \mathbb{E}_{s_{1:t}, a_{1:t-1}} \left[ \mathbb{E}_{a_t | s_t} \left[ \nabla_\theta \log \pi_\theta(a_t|s_t) b(s_t) \right] \right] \\
&= \mathbb{E}_{s_{1:t}, a_{1:t-1}} \left[ b(s_t) \mathbb{E}_{a_t | s_t} \left[ \nabla_\theta \log \pi_\theta(a_t|s_t) \right] \right] \\
&= 0
\end{aligned}
\end{equation*}

\end{document}
